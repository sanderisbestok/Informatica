Download


Source

PDF
Actions

   Copy Project
   Publish as Template
   Word Count
Sync

   Dropbox
   GitHub
   Mendeley
Settings

Compiler
Main document
Spell Check
Auto-Complete
Theme
Keybindings
Font Size
PDF Viewer
Hotkeys

   Show Hotkeys
Alco Sommen 2A
 main.tex


1
2
3
4
5
6
7
8
9
10
11
12
13
14
15
16
17
18
19
20
21
22
23
24
25
26
27
28
29
30
31
32
33
34
35
36
37
38
39
\documentclass{article}
\usepackage[utf8]{inputenc}
\usepackage{mathtools}
\usepackage{authblk}
\usepackage{comment}
\title{Sommen Serie 1}
\author{Nienke van den Nouwland (11016566); Sander Hansen (10995080); Ryan Blokker (10887881); Remy van der Vegt
(11013494)}
\date{\today}
\begin{document}
\maketitle
\begin{enumerate}
    \item
    \begin{enumerate}
        \item
        \item
    \end{enumerate}
    \item
    \item
    \begin{enumerate}
        \item Ja dit kan met beide algoritmen. En kan soms echter voor het overzicht fijn zijn om een bepaalde constante
        waarde aan alle kanten toe te voegen. Deze moet zo hoog zijn dat het kleinste getal positief wordt. Vervolgens kan
        je gaan zoeken naar de minimale opspannende boom. Om na het vinden van de oplossing deze constante weer van elke
        kant af te trekken. Dit is echter niet noodzakelijk.
        \item Om het maximale gewicht van een \textit{spanning tree} te vinden kunnen we het Kruskal algoritme 'omdraaien'.
        We doen dit als volgt. Allereerst kiezen we de kant met het maximale gewicht. Vervolgens ordenen we de kanten
        aflopend op gewicht. Dan voegen we de volgende kant (volgens onze geordende volgorde) toe aan de verzameling. Maar
        alleen als deze kant geen cykel vormt. Deze stap herhalen we net zo lang tot we alle knopen met elkaar
        hebben verbonden en we een maximale opspannende boom hebben gemaakt.
    \end{enumerate}
    \item
    \begin{enumerate}
        \item
        \item
        \item
    \end{enumerate}
\end{enumerate}
\end{document}


  Recompile
 Sommen Serie 1Nienke van den Nouwland (11016566); Sander Hansen (10995080);Ryan Blokker (10887881); Remy van der Vegt (11013494)September 21, 20161.   (a)(b)2.3.   (a)  Ja  dit  kan  met  beide  algoritmen.    En  kan  soms  echter  voor  hetoverzicht jn zijn om een bepaalde constante waarde aan alle kantentoe te voegen.  Deze moet zo hoog zijn dat het kleinste getal positiefwordt.  Vervolgens kan je gaan zoeken naar de minimale opspannendeboom.  Om na het vinden van de oplossing deze constante weer vanelke kant af te trekken.  Dit is echter niet noodzakelijk.(b)  Om het maximale gewicht van eenspanning treete vinden kunnen wehet Kruskal algoritme 'omdraaien'.  We doen dit als volgt.  Allereerstkiezen  we  de  kant  met  het  maximale  gewicht.   Vervolgens  ordenenwe de kanten a
opend op gewicht.  Dan voegen we de volgende kant(volgens  onze  geordende  volgorde)  toe  aan  de  verzameling.   Maaralleen  als  deze  kant  geen  cykel  vormt.   Deze  stap  herhalen  we  netzo lang tot we alle knopen met elkaar hebben verbonden en we eenmaximale opspannende boom hebben gemaakt.4.   (a)(b)(c)1
