%%%%%%%%%%%%%%%%%%%%%%%%%%%%%%
% LATEX-TEMPLATE TECHNISCH RAPPORT
%-------------------------------------------------------------------------------
% Voor informatie over het technisch rapport, zie
% http://practicumav.nl/onderzoeken/rapport.html
% Voor readme en meest recente versie van het template, zie
% https://gitlab-fnwi.uva.nl/informatica/LaTeX-template.git
%%%%%%%%%%%%%%%%%%%%%%%%%%%%%%

%-------------------------------------------------------------------------------
%	PACKAGES EN DOCUMENT CONFIGURATIE
%-------------------------------------------------------------------------------

\documentclass{uva-inf-article}
\usepackage[dutch]{babel}
\usepackage{listings}

% Relevant voor refereren vanaf blok 5
%\usepackage[style=authoryear-comp]{biblatex}
%\addbibresource{bib}

%-------------------------------------------------------------------------------
%	GEGEVENS VOOR IN DE TITEL
%-------------------------------------------------------------------------------

% Vul de naam van de opdracht in.
\assignment{Besturingssystemen}
% Vul het soort opdracht in.
\assignmenttype{Report}
% Vul de titel van de eindopdracht in.
\title{Myalloc}

% Vul de volledige namen van alle auteurs in.
\authors{Shakirullah Hamza; Sander Hansen}
% Vul de corresponderende UvAnetID's in.
\uvanetids{11061413; 10995080}

% Vul altijd de naam in van diegene die het nakijkt, tutor of docent.
\tutor{}
% Vul eventueel ook de naam van de docent of vakcoordinator toe.
\docent{}
% Vul hier de naam van de PAV-groep  in.
\group{}
% Vul de naam van de cursus in.
\course{}
% Te vinden op onder andere Datanose.
\courseid{}

% Dit is de datum die op het document komt te staan. Standaard is dat vandaag.
\date{\today}

%-------------------------------------------------------------------------------
%	VOORPAGINA
%-------------------------------------------------------------------------------

\begin{document}
\maketitle

%-------------------------------------------------------------------------------
%	INHOUDSOPGAVE EN ABSTRACT
%-------------------------------------------------------------------------------

% Niet doen bij korte verslagen en rapporten
%\tableofcontents
%\begin{abstract}
%\lipsum[13]
%\end{abstract}

%-------------------------------------------------------------------------------
%	INTRODUCTIE
%-------------------------------------------------------------------------------

\section{Introductie}
Bij deze opdracht voor besturingssystem was het doel om een eigen geheugen allocatie
functie te implementeren. In dit technische rapport zijn de doelen te vinden die
wij geprobeerd hebben te halen. Vervolgens is er een overzicht van welke datastructuren
en algoritmes wij gebruikt hebben.
\subsection{Doelen}
De doelen die wij geprobeerd hebben te halen zijn;
\begin{enumerate}
\item Auteurs bestand en report.pdf
\item Geen errors
\item Geen warnings
\item Werkende mmalloc
\item Werkende mmfree
\end{enumerate}

%-------------------------------------------------------------------------------
%	METHODE
%-------------------------------------------------------------------------------

\section{Methode}
\subsection{Datastructuren}
\subsubsection{Header}
De header structuur is een datastructuur die informatie bevat over het geheugen.
De voornaamste reden dat deze datastructuur is bestaat is omdat deze in een \textit{array}
bijhoud welke \textit{banks} actief zijn. Ook bevat het een getal dat aangeeft hoeveel
\textit{banks} er in totaal beschikbaar zijn in het geheugen. Het bevat ook nog twee
getallen die aangeven waar de header start en waar deze eindigd.
\begin{lstlisting}
struct mm_state {
    size_t startBank;
    size_t endBank;
    struct nodeInfo* next;
    size_t totalBanks;
    char activeBanks[];
};
\end{lstlisting}

\subsubsection{node}
Deze datastructuur bevat informatie over een bepaald stuk gealloceerd geheugen.
Zo bevat het twee getallen die aangeven bij welke \textit{bank} de \textit{node} start en bij welke
deze eindigd. Verder wijst deze \textit{node} naar een eventuele volgende \textit{node}
en bevat het een getal die aangeeft of de \textit{node} leeg is of nog gevuld.
\begin{lstlisting}
struct nodeInfo {
    size_t startBank;
    size_t endBank;
    struct nodeInfo* next;
    size_t empty;
};
\end{lstlisting}
\subsection{Algortimes}
\subsubsection{Initialisatie}
De initialisatie wordt eenmalig uitgevoerd om het geheugen te initialiseren. Tijdens de
initialisatie zullen een aantal gegevens over het geheugen worden opgeslagen. Vervolgens
zal deze datastructuur als \textit{return} waarde worden teruggegeven.

\begin{lstlisting}
hoeveelheidBanks = krijgRamInfo()

banksNodigVoorInit = ((grootteHeader + Bankgrootte - 1) +
	(charGrootte * hoeveelheidBanks)) / bankGrootte)

voor elke banksNodigVoorInit
	activeer bank

header.start = 0
header.volgende = NULL
header.einde = banksNodigVoorInit - 1
header.hoeveelheidBanks = hoeveelheidBanks

voor elke hoeveelheidBanks
    als banksNodigVoorInit
	   header.actieveBanks is 1
    anders
       header.actieveBanks is 0

return header
\end{lstlisting}
\subsubsection{Allocatie}
De allocatie wordt uitgevoerd als er geheugen gealloceerd moet worden. Deze functie wordt
aangeroepen met een \textit{header} en een bepaalde hoeveel \textit{bytes} die moeten worden
gealloceerd. Er wordt berekent hoeveel \textit{banks} er nodig zijn en vervolgens
wordt er gekeken of en waar deze hoeveelheid \textit{banks} in het geheugen geplaatst kan worden.
Hij returned een pointer met het geheugen adres.
\begin{lstlisting}
banksNodigVoorAlloc = ((grootteNode + hoeveelheidBytesNode + bankGrootte -1)
    / bankGrootte)

als header.volgende is NULL
    geheugenAdress = (basisAdress + header.einde + 1) * bankGrootte
    node = geheugenAdress
    node.start = header.einde + 1
    node.einde = node.start + banksNodigVoorAlloc -1
    node.volgende = NULL
    node.leeg = 0
    activeerBanks()

    return geheugenadress

huidige = header.volgende

zolang 1
    als huidige.leeg is 0
        als huidige.volgende is niet 0
            huidige = huidige.volgende
        anders
            als banksNodigVoorAlloc past in het geheugen
                geheugenAdress = (basisAdress + huidige.einde + 1) * bankGrootte
                node = geheugenAdress
                node.start = huidige.einde + 1
                node.einde = node.start + banksNodigVoorAlloc -1
                node.volgende = NULL
                node.leeg = 0
                activeerBanks()

                return geheugenadress
            anders
                return 0
    anders
        als banksNodigVoorAlloc past in de Node
            node.leeg = 1
            als nog ruimte over in Node
                geheugenAdress = (basisAdress + huidige.einde + 1) * bankGrootte
                node = geheugenAdress
                node.start = huidige.einde + 1
                node.einde = node.start + banksNodigVoorAlloc -1
                node.volgende = NULL
                node.leeg = 0
            return geheugenadress;
        anders
            als huidige.einde is totalBanks

            return 0

\end{lstlisting}

\subsubsection{Vrijgeven}
Het vrijgeven wordt aangeroepen wanneer een bepaald stuk geheugen niet meer gebruikt
wordt. Deze functie wordt aangeroepen met de \textit{header} en een \textit{pointer}
naar het adres waar dit geheugen zich bevindt.

\begin{lstlisting}
adres = (pointer - basisAdres - grootteVanNode) / bankGrootte

node = header.volgende

zolang 1
    als node.start is adres
        node.leeg = 1
        return
    anders
        node = node.volgende
\end{lstlisting}

%-------------------------------------------------------------------------------
%	REFERENTIES
%-------------------------------------------------------------------------------

%\printbibliography

%-------------------------------------------------------------------------------
%	BIJLAGEN EN EINDE
%-------------------------------------------------------------------------------

%\section{Bijlage A}
%\section{Bijlage B}
%\section{Bijlage C}
\end{document}
