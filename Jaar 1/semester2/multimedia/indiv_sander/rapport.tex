% LATEX-TEMPLATE TECHNISCH RAPPORT
%-------------------------------------------------------------------------------
% Voor informatie over het technisch rapport, zie
% http://practicumav.nl/onderzoeken/rapport.html
% Voor readme en meest recente versie van het template, zie
% https://gitlab-fnwi.uva.nl/informatica/LaTeX-template.git
%%%%%%%%%%%%%%%%%%%%%%%%%%%%%%

%-------------------------------------------------------------------------------
%	PACKAGES EN DOCUMENT CONFIGURATIE
%-------------------------------------------------------------------------------

\documentclass{uva-inf-article}
\usepackage[dutch]{babel}

% Relevant voor refereren vanaf blok 5
%\usepackage[style=authoryear-comp]{biblatex}
%\addbibresource{bib}

%-------------------------------------------------------------------------------
%	GEGEVENS VOOR IN DE TITEL
%-------------------------------------------------------------------------------

% Vul de naam van de opdracht in.
\assignment{Multimedia}
% Vul het soort opdracht in.
\assignmenttype{Rapport}
% Vul de titel van de eindopdracht in.
\title{Individueel verslag}

% Vul de volledige namen van alle auteurs in.
\authors{Sander Hansen}
% Vul de corresponderende UvAnetID's in.
\uvanetids{10995080}

% Vul altijd de naam in van diegene die het nakijkt, tutor of docent.
%\tutor{Naam van de tutor}
% Vul eventueel ook de naam van de docent of vakcoordinator toe.
%\docent{}
% Vul hier de naam van de PAV-groep  in.
%\group{Naam van de groep}
% Vul de naam van de cursus in.
%\course{Naam van de (gekoppelde) cursus}
% Te vinden op onder andere Datanose.
\courseid{}

% Dit is de datum die op het document komt te staan. Standaard is dat vandaag.
\date{\today}

%-------------------------------------------------------------------------------
%	VOORPAGINA
%-------------------------------------------------------------------------------

\begin{document}
\maketitle

%-------------------------------------------------------------------------------
%	INHOUDSOPGAVE EN ABSTRACT
%-------------------------------------------------------------------------------

% Niet doen bij korte verslagen en rapporten
%\tableofcontents
%\begin{abstract}
%\lipsum[13]
%\end{abstract}

%-------------------------------------------------------------------------------
%	INTRODUCTIE
%-------------------------------------------------------------------------------

\section{Inleiding}
Tijdens het laatste project van het eerstejaar informatica was de opdracht om
een android applicatie te maken die gebruik maakt van multimedia. Onze groep heeft er
voor gekozen om een \textit{multiplayer} spel te maken. Waarbij de twee toestellen
die tegen elkaar spelen met elkaar communiceren via de flits van hun camera.

\section{Persoonlijke bijdrage}
In principe was ik verantwoordelijk voor de media die in onze app gebruikt wordt.
Hierbij kan gedacht worden aan de afbeeldingen en muziek. Een van deze media onderdelen
was de achtergrond van het spel. De achtergrond moest een achtergrond zijn waarbij
het voor de speler lijkt alsof deze daadwerkelijk door het spel vliegt. Daarom is
er gekozen voor een brede achtergrond waarvan het einde op het begin aansluit.
Op deze manier kan het spel eindeloos gespeeld worden terwijl de achtergrond
beweegt en telkens weer op zichzelf aansluit.
\\\\Deze taak was echter vrij snel volbracht. Daarom was ik samen met Dennis ook
verantwoordelijk voor de beginpagina.
Toen we begonnen met het programmeren van de beginpagina maakten we alle losse
onderdelen telkens opnieuw aan, waarbij we ook de stijl elke keer moesten defini\"{e}ren.
Voor sommige van deze onderdelen was dit niet praktisch, gezien de stijl van bijvoorbeeld
de knoppen toch gelijk moest zijn. Daarom hebben we besloten om deze in de \textit{XML} bestanden
te verwerken. Dit maakte de code niet alleen overzichtelijker maar dit maakte het
ook makkelijker om alle knoppen in een keer aan te passen.
\\\\Ook zijn er een aantal beslissingen genomen waar we niet allemaal direct aan hebben
meegecodeerd maar wel aan hebben meegedacht. Zo wordt er gebruik gemaakt van
RenderScript, een \textit{framework} dat een taak over alle beschikbare processoren
verdeeld. Bij het afronden van het project kwamen we er ook achter dat het algoritme
dat bedacht was om te berekenen of er een cirkel aanwezig was, niet goed werkte.
Daarom hebben we besloten dit stuk code uit het RenderScript gedeelte te halen en
te vervangen door een functie van OpenCV. Dit was in het begin erg traag, maar relatief
eenvoudig op te lossen door de resolutie van de camera wat omlaag te schroeven.
%-------------------------------------------------------------------------------
%	BIJLAGEN EN EINDE
%-------------------------------------------------------------------------------

%\section{Bijlage A}
%\section{Bijlage B}
%\section{Bijlage C}
\end{document}
