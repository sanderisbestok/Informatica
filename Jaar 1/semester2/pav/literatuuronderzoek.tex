% LATEX-TEMPLATE LITERATUURONDERZOEK
%-------------------------------------------------------------------------------
% Voor informatie over het literatuuronderzoek, zie
% http://practicumav.nl/onderzoeken/literatuur.html
% Voor readme en meest recente versie van het template, zie
% https://gitlab-fnwi.uva.nl/informatica/LaTeX-template.git
%%%%%%%%%%%%%%%%%%%%%%%%%%%%%%

%-------------------------------------------------------------------------------
%	PACKAGES EN DOCUMENT CONFIGURATIE
%-------------------------------------------------------------------------------

\documentclass[12pt]{uva-inf-article}
\usepackage[dutch]{babel}

% Relevant voor refereren vanaf blok 5
\usepackage{cleveref}

%-------------------------------------------------------------------------------
%	GEGEVENS VOOR IN DE TITEL
%-------------------------------------------------------------------------------

% Vul de naam van de opdracht in.
\assignment{Virtualisatie}
% Vul het soort opdracht in.
\assignmenttype{Literatuuronderzoek}
% Vul de titel van de eindopdracht in.
\title{Virtualisatie en Virussen}

% Vul de volledige namen van alle auteurs in.
\authors{Sander Hansen}
% Vul de corresponderende UvAnetID's in.
\uvanetids{10995080}

% Vul altijd de naam in van diegene die het nakijkt, tutor of docent.
\tutor{Robin Klusman}
% Vul eventueel ook de naam van de docent of vakcoordinator toe.
\docent{}
% Vul hier de naam van de PAV-groep  in.
\group{A2}
% Vul de naam van de cursus in.
\course{Besturingssystemen}
% Te vinden op onder andere Datanose.
\courseid{}

% Dit is de datum die op het document komt te staan. Standaard is dat vandaag.
\date{\today}

%-------------------------------------------------------------------------------
%	VOORPAGINA
%-------------------------------------------------------------------------------

\begin{document}
\maketitle

%-------------------------------------------------------------------------------
%	INHOUDSOPGAVE EN ABSTRACT
%-------------------------------------------------------------------------------

% Niet doen bij korte verslagen en rapporten
%\tableofcontents
%\begin{abstract}
%\end{abstract}

%-------------------------------------------------------------------------------
%	INTRODUCTIE
%-------------------------------------------------------------------------------
\section{Inleiding}
In dit literatuuronderzoek zal er gekeken worden naar het gebruik van
virtualisatie om computers te beschermen tegen virussen. Het begrip virtualisatie,
dat nader zal worden toegelicht, biedt perspectief op het gebied van het beschermen
tegen virussen.

Omdat computers door bijna iedereen gebruikt worden tegenwoordig, hebben virussen
ook op vrijwel iedereen effect. Het is daarom belangrijk dat er onderzoek gedaan
wordt naar deze virussen, waarbij virtualisatie een steeds grotere rol gaat spelen.
De maatschappelijke relevantie is hier logischerwijs uit te concluderen. Dat ook
bedrijven hier mee bezig zijn toont het patent dat is aangevraagd op het beschermen
van anti-virus \textit{software} met behulp van virtualisatie aan.
\cite{wang2012securing}

Er is veel onderzoek gedaan naar wat virtualisatie kan beteken bij de bescherming
tegen virussen en onderzoek daar van. Openbare onderzoeken die de verschillende
aspecten hiervan toelichten en vergelijken zijn er echter nog niet. Daarom is het belangrijk om te kijken waar we
op dit moment staan met de bescherming van computers tegen virussen met behulp
van virtualisatie.

\subsection{Vraagstelling}
Omdat de maatschappelijke relevantie groot is en er veel onderzoek naar losse
onderdelen binnen dit gebied is gedaan, maar deze nog niet aan elkaar zijn gekoppeld.
Kunnen we de volgende vraag stellen;
\\\\Hoe kan virtualisatie gebruikt worden om computers te beschermen tegen virussen?

\subsection{Virtualisatie}
Allereerst is het belangrijk om te begrijpen wat virtualisatie inhoudt en wanneer
dit toegepast kan worden. Virtualisatie maakt het mogelijk om op dezelfde hardware
meerdere besturingssystemen tegelijkertijd te laten draaien.
Om virtualisatie mogelijk te maken kan er gebruik gemaakt worden van speciale
\textit{software} die de \textit{hardware} vanuit een hoofdbesturingssysteem
onderverdeeld over verschillende gastsystemen. Deze gastsystemen gedragen zich
dan meestal (afhankelijk van het soort virtualisatie) als autonome systemen.

Het is per definitie niet nodig om de \textit{hardware} van een computer aan te
passen om deze gebruik te laten maken van virtualisatie. Als de \textit{hardware}
van een bepaalde computer niet direct werkt met bepaalde virtualisatie \textit{software}
zal deze \textit{software} de code van de \textit{kernel} vertalen op binair niveau.
Vervolgens kan de virtualisatie alsnog draaien op de machine. \cite{adams2006comparison}

Het is wel zo dat om randapparatuur te virtualiseren er wel extra aanpassingen
nodig zijn. Echter wat voor aanpassingen hangt af van de virtualisatie \textit{software}
die gebruikt wordt. Het virtualiseren van randapparatuur kan vaak worden gerealiseerd
door het installeren van een aantal \textit{hardware} specifieke \textit{drivers}.
\cite{rossier2012embeddedxen}

Het is goed om te realiseren dat er meerdere soorten virtualisatie bestaan. Zo
bestaat er para virtualisatie, een vorm van virtualisatie waarbij het gastsysteem
zich ervan bewust is dat het gevirtualiseerd is. In plaats van de \textit{hardware}
direct aan te sturen zal het gastbesturingssysteem dit via het
hoofdbesturingssysteem laten gebeuren. Bij volle virtualisatie daarentegen gedraagt
het gastbesturingssysteem zich daadwerkelijk als autonoom systeem en stuurt het
direct zijn (gevirtualiseerde) \textit{hardware} aan.

Er bestaan verschillende \textit{software} pakketten voor virtualisatie. Er is niet
een bepaalde soort \textit{software} die duidelijk het meest gebruikt wordt maar
uitgaande van de namen die het meest voorkomen in onderzoeken lijken de volgende
drie bedrijven die virtualisatie \textit{software} maken het populairst; VMware,
Oracle en VirtualBox. Het is echter niet duidelijk op welke schaal ze door welke bedrijven
worden gebruikt, het blijft daarom gissen naar een antwoord.

%-------------------------------------------------------------------------------
%	KERN
%-------------------------------------------------------------------------------

% Geef de secties inhoudelijke titels die de lezer vertellen waar de sectie over
% gaat - dus niet 'kern'.
\section{Onderzoek naar virussen}
Het is lastig om onderzoek te doen naar virussen. De definitie van een computervirus
is volgens van Dale niet voor niets; '\textit{programma dat data ongevraagd kan veranderen
of vernietigen.}' \cite{vandale} Tijdens onderzoek naar deze virussen wordt er juist
data verzameld om te leren hoe deze \textit{software} precies werkt. Als deze data op een
of andere manier weer veranderd of vernietigd wordt heeft het onderzoek dus weinig
tot geen nut.

\subsection{Scheiden van systemen}
Virtualisatie kan het probleem dat bij virus onderzoek optreedt verhelpen.
Vanuit een hoofdbesturingssysteem kunnen gastbesturingssystemen worden waargenomen.
De gastsystemen kunnen worden
ge\"{i}nfecteerd met een bepaald virus terwijl het hoofdsysteem hiervoor niet
vatbaar is. Dit doordat virtualisatie er voor zorgt dat de twee systemen volledig
gescheiden zijn. Vervolgens kan het gastbesturingssysteem vanuit het hoofdbesturingssysteem
worden waargenomen.
Het uiteindelijke doel hiervan zal zijn om informatie over deze
virussen te vergaren en in de toekomst de virussen te slim af te zijn.


\subsection{Detecteren van virtualisatie}
Het onderzoek naar virussen wordt echter gehinderd door \textit{malware} dat
virtualisatie kan detecteren. Als er wordt ontdekt dat er gebruik wordt gemaakt
van virtualisatie kan de kwaadaardige \textit{software} het onderzoek verstoren
en op deze manier de onderzoeker verwarren.

Het ontdekken van virtualisatie kan bijvoorbeeld gebeuren door te zoeken naar
communicatie kanalen tussen het hoofd- en het gastbesturingssysteem. Als deze
worden gevonden duidt dit op virtualisatie. Vervolgens kan het virus gaan interfereren
op dit communicatie kanaal waardoor er dus verkeerde gegevens worden verstuurd.

Het is daarom ook belangrijk om er voor te zorgen dat het virus de virtualisatie niet
kan detecteren. De manier van verhullen is echter wel afhankelijk van de manier
waarop deze geprobeerd wordt te ontdekken. De bovenstaande manier is volgens
het onderzoek van Carpenter de meest gebruikte manier. \cite{carpenter2007hiding}

Het verhullen van deze manier kan gedaan worden door in de \textit{software} die
voor de virtualisatie zorgt een aantal aanpassingen in de instellingen te doen.
Het gaat er hierbij om dat door de instellingen te veranderen, de taal waarmee
de besturingssystemen met elkaar communiceren gewijzigd wordt. Hierdoor herkennen
de virussen niet meer dat er gecommuniceerd wordt tussen hoofd- en gastbesturingssystemen
en detecteren ze niet meer dat er gebruik gemaakt wordt van virtualisatie. Op die manier kan er
dus betrouwbaarder onderzoek gedaan worden.

\subsection{Deel conclusie}
Virtualisatie kan systemen scheiden waardoor er op een veilige manier onderzoek
naar virussen gedaan kan worden. Er moet om betrouwbare resultaten uit het onderzoek te krijgen wel gezorgd worden
dat het virus de resultaten niet be\"{i}nvloed. Dit kan gedaan worden door de
virtualisatie te verbergen.


VANAF HIER NAKIJKEN

\section{Virtualisatie en virusscanners}
Virusscanners zijn er voor ontworpen om computers zo goed mogelijk te beschermen
tegen virussen. Het probleem is echter vaak dat zodra \textit{malware} een computer heeft
ge\"{i}nfecteerd deze de virusscanner zal proberen uit te schakelen. Virtualisatie
biedt wellicht een uitkomst voor dit probleem. Door de virusscanner op een apart
virtueel systeem te laten draaien is deze niet meer vatbaar voor het virus.

\subsection{Virtual Machine Introspection tools}
\textit{Virtual Machine Introspection} tools afgekort VMI tools, zijn tools die
er ervoor zorgen dat een virtuele machine van buiten af kan worden waargenomen.
Met behulp van deze tools kan een virusscanner het systeem dus in de gaten houden
zonder dat het zelf gevaar loopt om ge\"{i}nfecteerd te worden.
\cite{garfinkel2003virtual}

Er zijn veel verschillende manieren waarop een VMI tool kan werken, zo kan de
tool slechts waarnemen en eventuele problemen doorgeven aan het besturingssystemen,
maar het zou ook direct kunnen ingrijpen.
Het detecteren van eventuele virussen kan ook op verschillende manieren gebeuren,
als voorbeeld kunnen de \textit{hash} tabellen van de instructie- en geheugen pagina's
worden vergeleken. Mocht hier geen match tussen zijn dan is de instructie pagina
waarschijnlijk corrupt.
\cite{nance2008virtual}

\subsection{Actief waarnemen}
Het probleem van de tools die gebruikt worden om de virtuele machines met elkaar
te verbinden is dat deze afhankelijk zijn van passief waarnemen. Dit houdt in
dat de virusscanners pas achteraf en niet preventief kunnen ingrijpen. Als er
alleen maar passief waargenomen zou kunnen worden zou er geen toekomst zijn in de verdere
ontwikkeling van een volledig anti-virus systeem dat draait op een virtuele machine.

In plaats van het systeem passief waar te nemen kan dit ook actief gebeuren.
Met het actief waarnemen van een systeem wordt bedoelt dat er via een koppeling
actief naar het waar te nemen systeem wordt gekeken en dat er op die manier ook
preventief gehandeld kan worden.

Via deze koppeling kunnen handelingen die de computer uitvoert worden bekeken.
Bij deze handelingen kan er gedacht worden aan het aan maken van processen, het
schrijven naar een harde schijf en dergelijke. Deze processen kunnen dan beveiligd
worden en verdachte activiteiten, die op \textit{malware} zouden kunnen duiden,
kunnen worden herkent. \cite{payne2008lares}

\subsection{Deel conclusie}
Virtuele systemen zijn uiterst geschikt om virusscanners veiliger te maken. Door
deze op een apart systeem te laten draaien zijn ze niet vatbaar voor de virussen
zelf. Als er een koppeling tussen het systeem van de virusscanner en het te observeren
systeem wordt geplaatst kunnen deze virusscanner op dezelfde manier werken als
normale virusscanners.


%-------------------------------------------------------------------------------
%	DISCUSSIE
%-------------------------------------------------------------------------------
\section{Discussie}
In dit onderzoek is er gekeken naar op welke verschillende manieren virtualisatie
gebruikt kan worden bij het beschermen van computers tegen virussen. Het is
gebleken dat virtualisatie op meerdere manieren kan worden ingezet bij het
beschermen van computers tegen virussen.

Helaas zijn lang niet alle onderzoeken naar het gebruik van virtualisatie
bij het beschermen van computers tegen virussen openbaar. Dit komt waarschijnlijk omdat
er veel geld te verdienen is met bijvoorbeeld virusscanners en daarom dus ook
met onderzoek wat betrekking heeft op virtualisatie en virussen. Dat is dan ook
rede waarom er veel patenten op dit gebied zijn.

\subsection{Conclusie}
Virtualisatie biedt een veilige omgeving waarin systemen elkaar niet kunnen
infecteren. Daardoor kan virtualisatie zowel bij onderzoek naar virussen als bij
het daadwerkelijk verhelpen en beschermen tegen virussen worden gebruikt.

%-------------------------------------------------------------------------------
%	REFERENTIES
%-------------------------------------------------------------------------------
\bibliography{references}
\bibliographystyle{plainnat}

%-------------------------------------------------------------------------------
%	BIJLAGEN EN EINDE
%-------------------------------------------------------------------------------
%\section{Bijlage A}
%\section{Bijlage B}
%\section{Bijlage C}

\end{document}
