\documentclass[a4paper]{article}
\usepackage[dutch]{babel}
\usepackage{longtable}
%\usepackage[style=authoryear-comp]{biblatex}
\usepackage{cleveref}
\begin{document}


\begin{longtable}{  p{35pt} p{300pt} p{50pt} }
    \caption{Hoe kan virtualisatie gebruikt worden om computers te beveiligen??}\\
    \textbf{Sectie} &  \textbf{Welke informatie?} &  \textbf{Bronnen} \\ \hline

    Inleiding 	& \textit{Maatschappelijke relevantie: Als virtuaagelisatie een mogelijkheid biedt om computers
                te beveiligen tegen computervirussen dan kan dit een uitkomst zijn
                om computers over het algemeen veiliger te maken.} & \cite{heiser2008role} \\ \cline{2-3}
    		&  \textit{Eerdere bevindingen: In hoofdlijnen is er bekend over virtualisatie dat het zijn virtuele systemen
            gescheiden houdt. Doordat de systemen gescheiden zijn is er geen gevaar dat
            een andere virtueel systeem dat op dezelfde machine draait virussen over draagt naar andere systemen.}& \\ \cline{2-3}
		&  \textit{Wetenschappelijke relevantie: Combineren van verschillende beveiligings taktieken die gebruik
                                                maken van virtualisatie.} & \\ \cline{2-3}
		&  \textit{Centrale vraag: Hoe kan virtualisatie gebruikt worden om computers beter te beveiligen.}& \\ \cline{2-3}
		&  \textit{Opbouw verslag: Hoe zorgt virtualisatie ervoor dat systemen gescheiden blijven
                                   en vervolgens hoe kan dit gebruikt worden om computers beter te beveiligen.} & \\ \hline

     Kern	& \textit{Gescheiden systemen:} &  \\ \cline{2-3}
		& \textit{Deelvraag 1: Hoe houdt virtualisatie systemen gescheiden} & \cite{heiser2008role} \\ \cline{2-3}
        & \textit{Deelonderzoek1a: Wanneer kan je virtualisatie toepassen. (Welke soft- en hardware aanpassingen zijn hier voor nodig.)} & \cite{adams2006comparison} \\ \cline{2-3}
		& \textit{Deelonderzoek1b: Virtuele systemen onder elkaar (Hypervisor)} &  \\ \cline{2-3}
		& \textit{Deelonderzoek1c: Virtuele systemen naast elkaar} &  \\ \cline{2-3}
		& \textit{Deelconclusie 1: Onderzoek b en c lijken erg op elkaar, dit kan wellicht tot een deelonderzoek verwerkt worden.} &  \\ \cline{2-3}

		& \textit{Beveiliging:} &  \\ \cline{2-3}
		& \textit{Deelvraag 2: Hoe kan virtualisatie gebruikt worden om te beveiligen} &  \\ \cline{2-3}
		& \textit{Deelonderzoek 2a: Hoe kunnen virusscanners gebruik maken van virtualisatie?
                Het patent van dat Google heeft aangevraagd om virtualisatie
                te gebruiken om anti-virus software te beveiligen.} & \cite{wang2012securing} \\ \cline{2-3}
		& \textit{Deelonderzoek2b: Hoe kan je systemen tegen elkaar beveiligen ook al zijn ze met het zelfde netwerk verbonden?} & \cite{garber2012challenges} \\ \cline{2-3}
		& \textit{Deelconclusie 2: Door een virusscanner op een virtueel systeem te laten draaien
                                is deze zelf niet vatbaar voor virussen. Maar als twee systemen met elkaar verbonden zijn via een netwerk, is er extra
                                aandacht nodig voor de beveiliging en scheiding tussen deze twee systemen.} &  \\ \hline

   Discussie	& \textit{Deelconclusie 1 en 2:} &  \\ \cline{2-3}
		& \textit{Eindconclusie: Virtualisatie kan op verschillende manieren gebruikt worden om computers beter te beveiligen. Toch
        moet er niet direct vanuit worden gegaan dat virtualisatie per definitie veilig is omdat deze machines vaak binnen
        het netwerk met elkaar verbonden zijn.} &  \\ \cline{2-3}
		& \textit{Evaluatie en verklaringen: Dat virtualisatie systemen creeert die onafhankelijk van elkaar zijn is algemeen bekend} &  \\ \cline{2-3}
		& \textit{Terugkoppeling eerdere bevindingen: Zoals al bekend was houdt virtualisatie systemen gescheiden waardoor dit gebruikt kan worden voor beveiligings oplossingen } &  \\ \cline{2-3}
                     & \textit{Terugkoppeling brede context en maatschappelijke relevantie: Er wordt door bedrijven geinvesteerd in patenten op beveilingstechnieken met behulp van virtualisatie.} & \cite{wang2012securing} \\ \hline

\end{longtable}

\bibliography{authors}
\bibliographystyle{plainnat}

\end{document}
